% Copyright 2004 by Till Tantau <tantau@users.sourceforge.net>.
%
% In principle, this file can be redistributed and/or modified under
% the terms of the GNU Public License, version 2.
%
% However, this file is supposed to be a template to be modified
% for your own needs. For this reason, if you use this file as a
% template and not specifically distribute it as part of a another
% package/program, I grant the extra permission to freely copy and
% modify this file as you see fit and even to delete this copyright
% notice. 

\documentclass{beamer}
\usepackage[utf8]{inputenc}
\usepackage[serbian]{babel}
\usepackage{fancyvrb}

% There are many different themes available for Beamer. A comprehensive
% list with examples is given here:
% http://deic.uab.es/~iblanes/beamer_gallery/index_by_theme.html
% You can uncomment the themes below if you would like to use a different
% one:
%\usetheme{AnnArbor}
%\usetheme{Antibes}
%\usetheme{Bergen}
%\usetheme{Berkeley}
%\usetheme{Berlin}
%\usetheme{Boadilla}
%\usetheme{boxes}
%\usetheme{CambridgeUS}
%\usetheme{Copenhagen}
%\usetheme{Darmstadt}
%\usetheme{default}
%\usetheme{Frankfurt}
%\usetheme{Goettingen}
%\usetheme{Hannover}
%\usetheme{Ilmenau}
%\usetheme{JuanLesPins}
%\usetheme{Luebeck}
\usetheme{Madrid}
%\usetheme{Malmoe}
%\usetheme{Marburg}
%\usetheme{Montpellier}
%\usetheme{PaloAlto}
%\usetheme{Pittsburgh}
%\usetheme{Rochester}
%\usetheme{Singapore}
%\usetheme{Szeged}
%\usetheme{Warsaw}

\title{Napadi na softverske sisteme i mehanizmi zaštite}

% A subtitle is optional and this may be deleted
%\subtitle{Optional Subtitle}

\author{Čedomir Dimić, Jana Milutinović, Miloš Samardžija}
% - Give the names in the same order as the appear in the paper.
% - Use the \inst{?} command only if the authors have different
%   affiliation.

\institute{Matematički fakultet} % (optional, but mostly needed)

% - Use the \inst command only if there are several affiliations.
% - Keep it simple, no one is interested in your street address.

\date{maj 2017.}
% - Either use conference name or its abbreviation.
% - Not really informative to the audience, more for people (including
%   yourself) who are reading the slides online

% This is only inserted into the PDF information catalog. Can be left
% out. 

% If you have a file called "university-logo-filename.xxx", where xxx
% is a graphic format that can be processed by latex or pdflatex,
% resp., then you can add a logo as follows:

% \pgfdeclareimage[height=0.5cm]{university-logo}{university-logo-filename}
% \logo{\pgfuseimage{university-logo}}

% Delete this, if you do not want the table of contents to pop up at
% the beginning of each subsection:

% Let's get started
\begin{document}

\begin{frame}
  \titlepage
\end{frame}

\begin{frame}{Pregled}
  \tableofcontents
  % You might wish to add the option [pausesections]
\end{frame}

% Section and subsections will appear in the presentation overview
% and table of contents.
\section{Uvod}


\begin{frame}{Uvod}
  \begin{itemize}
  \item {
    Bezbednost sistema je jedna od najsloženijih tema u modernom računarstvu.
  }
  \pause
  \item {
    Povrede bezbednosti snose ogromne posledice najčešće u vidu novca. 
  }
  \pause
  \item {
    Stoga, veoma važan zadatak organizacije je da dobro zaštiti poverljive informacije kako one ne bi bile ukradene ili zloupotrebljene.
  }
  \pause
    \begin{block}{Definicija}
      Sigurnost softvera obuhvata razvoj i implementaciju softvera tako da se on zaštiti od zlonamernih napada i drugih bezbednosnih rizika, ali da istovremeno može da nastavi neometano da radi i pored tih rizika pritom zadržavajući sve predviđene funkcionalnosti.
    \end{block}
  \end{itemize}
\end{frame}

\subsection{CIA model}

% You can reveal the parts of a slide one at a time
% with the \pause command:
\begin{frame}{CIA model}
  \begin{itemize}
  \item {
    Kao 3 najvažnija aspekta sigurnosti softvera se smatraju poverljivost, integritet i raspoloživost podataka.
    %\pause % The slide will pause after showing the first item
    \pause
  }
   \item {
    Mere koje se preduzimaju da bi se obezbedila poverljivost su napravljene tako da se onemogući da se osetljive informacije nađu u pogrešnim rukama.
    %\pause % The slide will pause after showing the first item
    \pause
  }
  \item {   
    Integritet uključuje održavanje konzistentnosti, preciznosti i pouzdanosti podataka tokom njihovog čitavog životnog ciklusa.
    \pause
  }
  \item {   
    Raspoloživost se postiže održavanjem hardvera, preduzimanjem potrebnih popravki hardvera odmah kada za tim postoji potreba i održavanjem korektnog funkcionisanja operativnog sistema.
  }
  \end{itemize}
\end{frame}

\section{Mehanizmi zaštite sistema}

\begin{frame}{Mehanizmi zaštite sistema}
  \begin{itemize}
  \item {
   Osnovni mehanizmi zaštite na Internetu su: zaštitni zid(eng.firewall),antivirusni programi i šifrovanje podataka.
   \pause
  }
  \item {
    Zaštitni zid je mrežni sistem zaštite koji se koristi za praćenje i kontrolisanje dolazećeg i odlazećeg mrežnog saobraćaja.
    \pause
  }
  \item {
    Antivirusni programi sprečavaju da na računar dospe zlonamerni softver(eng. malware) ili ga uklanjaju po dospeću na računar.
    \pause
  }
  \item {
    U savremenom poslovanju mora postojati mehanizam koji obezbeđuje: zaštitu tajnosti informacija (sprečavanje otkrivanja njihovog sadržaja), integritet informacija (sprečavanje neovlašćene izmene informacija) i autentičnost informacija (definisanje i proveru identiteta pošiljaoca).
  }
  \end{itemize}
\end{frame}

\section{Napadi}

\begin{frame}{Napadi}
    \begin{itemize}
        \item Programeri se uglavnom fokusiraju na korektnost softvera, odnosno na postizanje željenog ponašanja softvera. \pause
        \item Bezbednost se odnosi na sprečavanje neželjenog ponašanja. \pause
        \item Problemi u dizajnu i implementaciji potencijalno mogu da učine aplikaciju ranjivom. \pause
        \item Pojava bagova je neizbežna. \pause
        \item Hakeri aktivno rade na pronalaženju bagova, koje kasnije mogu da iskoriste u svoje svrhe. \pause
        \item Potrebno je eliminisati sve propuste u dizajnu i implementaciji, ili barem otežati ili u potpunosti onemogućiti njihovo iskorišćavanje.
    \end{itemize}
\end{frame}

\subsection{Najčešći napadi}

\begin{frame}{Najčešći napadi}
    \begin{itemize}
      \item Prekoračenje bafera
        \begin{itemize}
            \item Bag koji pogađa programe pisane na programskom jeziku nižeg nivoa (uglavnom C i C++) \pause
            \item Pisanjem izvan granica bafera se može izazvati
            \begin{itemize}
                \item oštećenje podataka \pause
                \item nasilno zatvaranje programa \pause
                \item izvršavanje zlonamernog koda \pause
            \end{itemize}
            \item Pored ovog napada, postoje i varijacije
            \begin{itemize}
                \item prekoračenje hipa (eng.~{\em heap overflow})
                \item prekoračenje celog broja (eng.~{\em integer overflow})
                \item prekoračenje čitanjem (eng.~{\em read overflow})
            \end{itemize}
        \end{itemize}
    \end{itemize}
\end{frame}

\begin{frame}{Najčešći napadi}
    \begin{itemize}
        \item Podmetanje SQL upita
        \begin{itemize}
            \item Napad omogućava izvršavanje zlonamernih SQL upita. \pause
            \item Ranjivi su delovi aplikacije u kojima korisnički unos direktno učestvuje u upitu, bez prethodne obrade. \pause
            \item Iskorišćavanjem ovog propusta, napadač može da:
            \begin{itemize}
                \item zaobiđe mehanizme autentikacije i autorizacije \pause
                \item pročita sadržaj baze podataka \pause
                \item dodaje, modifikuje i briše zapise \pause
            \end{itemize}
            \item Prevencija napada se vrši upotrebom pripremljenih upita (eng.~{\em prepared statements}).\pause
        \end{itemize}
        \item Krađa sesije
        \begin{itemize}
            \item Predstavlja jedan od načina zloupotrebe kolačića (eng.~{\em cookies}). \pause
            \item Krađom identifikacionog kolačića, napadač se predstavlja kao autentikovan korisnik, i izvršava zlonamerne akcije u njegovo ime. \pause
            \item Napad se može sprečiti instalacijom dodataka za pregledače isključivo iz proverenih izvora, korišćenjem bezbedne veze, upotrebom skrivenih kolačića...
        \end{itemize}
    \end{itemize}
\end{frame}

\begin{frame}{Najčešći napadi}
    \begin{itemize}
        \item CSRF (eng.~{\em cross-site request forgery})
        \begin{itemize}
            \item Zlonamerna osoba ne mora da bude autentikovana da bi izvršila prevaru, već korisnike navodi da sami izvrše nepoželjnu akciju. \pause
            \item Mete napada su zahtevi koji vrše izmenu stanja aplikacije. \pause
            \item Napad se sprečava upotrebom CSRF tokena. \pause
        \end{itemize}
        \item XSS (eng.~{\em cross-site scripting})
        \begin{itemize}
            \item Napad zasnovan na umetanju koda. \pause
            \item Korisnički unos se tretira kao validan kod od strane JavaScript interpretera. \pause
            \item Postoje:
            \begin{itemize}
                \item \textbf{postojani XSS} (eng.~{\em persistent XSS}), gde maliciozni kod potiče iz baze podataka \pause
                \item \textbf{reflektovani XSS} (eng.~{\em reflected XSS}), gde maliciozni kod potiče iz zahteva žrtve (npr. URL) \pause
                \item \textbf{DOM-zasnovani XSS} (eng.~{\em DOM-based XSS}), gde je ranjivost na klijentskoj strani \pause
            \end{itemize}
            \item Prevencija se vrši enkodiranjem korisničkog unosa, i izbegavanjem umetanja unosa direktno u script tagove, u atribute za rukovaoce događajima (event handlers), ili u CSS.
        \end{itemize}
    \end{itemize}
\end{frame}

\section{Bezbednost i Veb serveri}

\begin{frame}{Bezbednost i Veb serveri}
    \begin{itemize}
        \item {
        Veb server (eng.~{\em Web Server}) je računar koji je odgovoran za preuzimanje i opsluživanje Veb stranica koje zahteva klijent
        } \pause
        \item{ VS obrađuje klijentske zahteve preko HTTP mrežnog protokola čiji je zadatak da distribuira informacije na Internetu}  \pause
        \item { Povezan je na Internet, te korisnici mogu da pristupe podacima koje server skladišti sa bilo kog mesta na Internetu, što može imati za posledicu pokušaje neovlašćenog pristupa i zloupotrebu podataka } \pause
         \item {Apache Web Server, Internet Information Server (IIS) , lighttpd , Sun Java System Web Server, Jigsaw Server }
    \end{itemize}
\end{frame}

\begin{frame}{Bezbednosni propusti pri implementaciji Veb servera }
    \begin{itemize}
        \item {Kontrola pristupa 
        \begin{itemize}
            \item  {\textbf{Kontrolom pristupa} (eng.~{\em Access Control}) reguliše se ko ima mogućnost pretraživanja i izvršavanja (CGI skripti) na serveru } \pause
        \item  { 
         Kontrola \textbf{čitanja} - štiti se poverljivost informacija } \pause
         \item{Kontrola \textbf{pristupa} - štiti se  integritet podataka } \pause
         \item{Obavezna dobra kontrola pristupa konfiguracijskih fajlova}  \pause
         \item{Kontrola pristupa smanjuje mogucnost otkrivanja osetljivih informacija koje ne smeju biti javno izložene}  \pause
         \item{ Dobra kontrola pristupa  će ograničiti koriščcenje resursa u slučaju napada } \pause
         \item{Primarni uređaji za kontrolu pristupa su ruter i zaštitni zid
        }
        \end{itemize}
        }
    \end{itemize}
\end{frame}

\begin{frame}{Bezbednosni propusti pri implementaciji Veb servera }
    \begin{itemize}
        \item {Bagovi u CGI skriptama 
        \begin{itemize}
            \item  {\textbf{CGI} (eng.~{\em Common Gateway Interface}) skripte su programi koji se na Veb serverima izvršavaju u realnom vremenu i čija je uloga da rukuju ulaznim podacima korisnika, pristupaju bazi i vraćaju informacije korisniku }
            \pause
            \item {Oprez: Korisnicki ulazni podaci mogu biti komande koje se automatski izvršavaju čime mogu da nanesu štetu host mašini i ugroze sigurnost servera}
            \pause
        \end{itemize}
        }
     \item { Mehanizmi logovanja 
        \begin{itemize}
            \item  {\textbf{Logovi } su dnevnici u kojima se cuvaju podaci o tome ko je i kada modifikovao, dodavao i pristupao serverskim komponenatama}\pause
        \item  { 
         Logovi su često i jedini pokazatelji sumnjivih aktivnosti }\pause
         \item{Dodatno - mehanizmi alarmiranja}\pause
         \item{praviti i rezervne kopije ( eng.~{\em  backup}) logova}
        \end{itemize}
        
        }
    \end{itemize}
\end{frame}



% Placing a * after \section means it will not show in the
% outline or table of contents.
\section{Zaključak}

\begin{frame}{Zaključak}
  \begin{itemize}
  \item
   I pored svih mehanizama zaštite koji postoje, softver nikada ne može biti u potpunosti bezbedan 
  \item
        S obzirom da su napadi na softverske sisteme učestaliji, mehanizmi zaštite se konstantno unapređuju 
  \item    
   Ovaj rad može predstavljati dobar uvod u detaljnije izučavanje pojedinih aspakata bezbednosti softvera koji su prikazani
  \end{itemize}
\end{frame}



\section<presentation>{Literatura} 
\begin{frame}
  \frametitle<presentation>{Literatura}
    
  \begin{thebibliography}{10}

  \bibitem{}
    Software Security
     \newblock  College Park University of Maryland.
    \newblock {\url{https://www.coursera.org/learn/software-security }}
   
    \bibitem{}
   Guidelines on Securing Public Web Server
    \newblock
     \newblock{\url{http://csrc.nist.gov/publications/nistpubs/800-44-ver2/SP800-44v2.pdf}}
   
 
  % Followed by interesting articles. Keep the list short. 

  \bibitem{}
    Avoiding the top 10 software security design flaws
    \newblock IEEE Computer Society
    \newblock  \url{www.computer.org/cms/CYBSI/docs/Top-10-Flaws.pdf}
   
  \end{thebibliography}
\end{frame}

\end{document}

