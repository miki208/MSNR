
% !TEX encoding = UTF-8 Unicode

\documentclass[a4paper]{article}

\usepackage[T2A]{fontenc} % enable Cyrillic fonts
\usepackage[utf8x,utf8]{inputenc} % make weird characters work
\usepackage[serbian]{babel}
%\usepackage[english,serbianc]{babel}
\usepackage{amssymb}

\usepackage{color}
\usepackage{url}
\usepackage[unicode]{hyperref}
\hypersetup{colorlinks,citecolor=green,filecolor=green,linkcolor=blue,urlcolor=blue}

\newcommand{\odgovor}[1]{\textcolor{blue}{#1}}

\begin{document}

\title{Pravne i etičke obaveze u svetu informacionih tehnologija\\ \small{Nikola, Miloš}}

%%%%%%%%%%%%%%%%%%%%%%%%%%%%%%%%%%%%%%%%%%%%%%%%%%%%%%%%%%%%%%%%%%%%%%%%%%%%%%%%%%%%%%%%%%
\author{Recenzija: Miloš Samardžija}
%ime autora recenzije neće biti predato autorima seminarskog rada
%%%%%%%%%%%%%%%%%%%%%%%%%%%%%%%%%%%%%%%%%%%%%%%%%%%%%%%%%%%%%%%%%%%%%%%%%%%%%%%%%%%%%%%%%%


\maketitle

%Recenziju predajete u tex obliku, budite uredni!
%Pod komentarima su data objašnjenja za svaku navedenu stavku.

%!!!ne brisite narednu liniju!!!
%pocetak teksta koji se predaje recenzentima


\section{O čemu rad govori?}
% Напишете један кратак пасус у којим ћете својим речима препричати суштину рада (и тиме показати да сте рад пажљиво прочитали и разумели). Обим од 200 до 400 карактера.
Rad definiše pojam etike u svetu informacionih tehnologija. Razmatra se zašto je pitanje etičkog ponašanja u informacionim tehnologijama jako bitno, i zašto je bitno unapređivati zakone vezane za informacione tehnologije. Opisani su problemi koji se javljaju u pokušaju da se unaprede takvi zakoni. Obrađene su tri oblasti gde konstantna promena informacionih tehnologija drastično utiče na zakon.

\section{Krupne primedbe i sugestije}
% Напишете своја запажања и конструктивне идеје шта у раду недостаје и шта би требало да се промени-измени-дода-одузме да би рад био квалитетнији.
Ne postoje krupnije primedbe i sugestije.

\section{Sitne primedbe}
% Напишете своја запажања на тему штампарских-стилских-језичких грешки
U nastavku su navedene predložene izmene delova teksta koji sadrže greške. Reči koje je potrebno ispraviti su podebljane.
\begin{enumerate}
	\item Uvod, prvi pasus, treća rečenica: \textit{O etičkim \textbf{pitanjim} u svetu informacionih tehonologija nije lako govoriti, pa ipak, etičko ponašanje se očekuje od profesionalaca u gotovo svim granama industrije.}
	\item Uvod, drugi pasus, druga rečenica: \textit{Pravni odgovor na ovo pitanje ne postoji, jer kako se svet tehnologije razvija i napreduje  veoma brzo, izrazito je teško sprovesti zakone i doneti ispravne odluke po pitanju morala i etike u svetu \textbf{informaiconih} tehnologija.}
	\item Osnovna pravna pitanja vezana za IT, prvi pasus, druga rečenica: \textit{Razvijanjem tehnologije konstantno je potrebno unapređivati zakone vezane za \textbf{informacijone} tehnologije, ali zbog jako brze evolucije zakon jednostavno ne stiže da se unapredi dovoljnom brzinom.}
	\item Elektronska trgovina, prvi pasus, treća rečenica: \textit{Kako je mnogo teže \textbf{utvrdidi} prepravljanje digitalnog dokumenta naspram fizičkog papirnog dokumenta na kome se svaka promena jasno vidi, potrebno je razviti način za kvalitetnu autentifikaciju i validaciju digitalnih dokumenata.}
	\item Elektronska trgovina, drugi pasus, druga i treća rečenica: \textit{Kada dodje do \textbf{trasnakcije} ko je nadležan? Ko je zadužen da održava zakon kod ove transkacije?}
	\item Zaštita privatnosti i podataka, prvi pasus, druga rečenica: \textit{Samim tim došlo je do porasta potražnje ličnih, \textbf{prihvatnih}, podataka te su se otvorile i firme koje kao jedini cilj imaju prikupljanje podataka.}
	\item Zaštita privatnosti i podataka, drugi pasus, treća rečenica: \textit{U Austriji, Danskoj, Kanadi, Francuskoj, Nemačkoj, Luksemburgu, Norveškoj, \textbf{Švetskoj} i Sjedinjenim Američkim Državama je zakon usvojen, dok je u Belgiji, Islandu, Španiji, Švajcarskoj i Holandiji pripremljen predlog zakona.}
\end{enumerate}

\section{Provera sadržajnosti i forme seminarskog rada}
% Oдговорите на следећа питања --- уз сваки одговор дати и образложење

\begin{enumerate}
\item Da li rad dobro odgovara na zadatu temu?\\
Ovaj rad odgovara na sva pitanja predviđena ovom temom.
\item Da li je nešto važno propušteno?\\
Nema važnijih stvari koje su propuštene.
\item Da li ima suštinskih grešaka i propusta?\\
Rad sadrži greške nastale prilikom kucanja, koje su naznačene u prethodnoj sekciji.
\item Da li je naslov rada dobro izabran?\\
Naslov rada je u skladu sa sadržajem rada.
\item Da li sažetak sadrži prave podatke o radu?\\
Sažetak sadrži prave podatke o radu.
\item Da li je rad lak-težak za čitanje?\\
Rad je lak za čitanje, i pruženi su primeri koji dodatno olakšavaju razumevanje obrađenih tema.
\item Da li je za razumevanje teksta potrebno predznanje i u kolikoj meri?\\
Za razumevanje teksta nije potrebno predznanje.
\item Da li je u radu navedena odgovarajuća literatura?\\
U radu je navedena odgovarajuća literatura, i citirana je u okviru rada.
\item Da li su u radu reference korektno navedene?\\
Reference su korektno navedene.
\item Da li je struktura rada adekvatna?\\
Rad sadrži naslov, sažetak, sadržaj, uvod, razradu, zaključak i literaturu. Struktura rada je adekvatna.
\item Da li rad sadrži sve elemente propisane uslovom seminarskog rada (slike, tabele, broj strana...)?\\
Rad sadrži slike, tabele, i ograničenja koja se odnose na broj strana i literaturu su zadovoljena.
\item Da li su slike i tabele funkcionalne i adekvatne?\\
Slike i tabele su funkcionalne i adekvatne, i referisane su u okviru teksta.
\end{enumerate}

\section{Ocenite sebe}
% Napišite koliko ste upućeni u oblast koju recenzirate: 
% a) ekspert u datoj oblasti
% b) veoma upućeni u oblast
% c) srednje upućeni
% d) malo upućeni 
% e) skoro neupućeni
% f) potpuno neupućeni
% Obrazložite svoju odluku
Srednje sam upućen u oblast koju recenziram. Sa ovom temom sam se susretao na nekim od prethodnih kurseva.

%kraj teksta koji se predaje recenzentima
%!!!ne brisite prethodnu liniju!!!

%%%%%%%%%%%%%%%%%%%%%%%%%%%%%%%%%%%%%%%%%%%%%%%%%%%%%%%%%%%%%%%%%%%%%%%%%%%%%%%%%%%%%%%%%%
\section{Poverljivi komentari}
% Poverljivi komentari neće biti prosleđeni autorima seminarskog rada.
% Ukoliko nemate poverljivih komentara, ovaj deo može da ostane prazan.
%%%%%%%%%%%%%%%%%%%%%%%%%%%%%%%%%%%%%%%%%%%%%%%%%%%%%%%%%%%%%%%%%%%%%%%%%%%%%%%%%%%%%%%%%%


\end{document}


